# Paper 02
Fraud Prediction in Smart Societies Using Logistic Regression and k‑fold Machine Learning Techniques
- Three types of Fraud 
    - No criminal history - for unguent need of money
    - criminal offenders - self defined banks 
    - organised crime - higher level of risk 


# Abstract
The trade fraud detection is an enormously difficult task. The main problem in executing trade fraud detection technique is the availability of limited amount of fraudulent company related data. In this research article, machine learning based fraud detection and prevention in trade credit has been presented. Different machine learning techniques like Random Forest, XDBoost, Artificial Neural network (ANN) are applied for fraudulent classification task. To improve model quality different method of handling class imbalanced are used. The implementation of proposed methodology and its further analysis using model selection tool like ROC (Receiver Operation Characteristic) curve, confusion matrix, mean-recall score value, and precision recall carves has been used. 

# Keywords
Fraud detection, Fraudulent, Trade Fraud, Confusion matrix, Machine learning, Mean-recall-score, ROC curve


# Introduction 
Because of increase in the trade fraud yearly [AMOUNT] financial losses happens. In today's era. the standard way for the companies to do transactions with other companies are based on trade credit. In real life it is not easy to track or analysis every company manually to mark as suspicious company. 
- The reason of trade fraud is increasing. 
- Why it is important - Financial loss for insurance company, suppliers may in bankruptcy state.


On the basis of types of trade fraud done happens, the researchers have divided the fraudulent types in three categories. buyers fraud, joint fraud and third party fraud. 

For this paper we have selected the third party company fraud. 

The trade fraud detection is a difficult task. The main problem in trade fraud detection is the availability of data related to fraudulent companies. Hence this enormously difficult tasks to find patterns [use the PCA image to show to complexity]. The company risk assessment teams use some of the known patterns based on internal and publicly available information. However the outline of normal and fraudulent company changes contently and overlaps a lot. Due to this its hard to create a suitable dataset. The researcher and fraud detection/ prevention agency synthetically generate data set. Also the trade fraud rated data is very skewed in nature where a fraudulent cases can be found after going though thousands of cases. 

The most frequent used machine learning techniques for detecting fraudulent are Support Vector Machines (SVM), Artificial Neural Networks (ANN), decision trees, logistic regression (LR), rule induction techniques, k-means cluster, genetic algorithms etc. These techniques can be applied separately or in a combined approach. 

# Related Work 
The suspicious company detection is very useful for financial institutions, suppliers and government. In today's world the number of fraud company has increased due to globalisation. The data mining technique based fraud detection/prevention models provide medium level of accuracy and reliability. 