\begin{abstract}
    Trade credit fraud is a major concern for suppliers and insurance companies. An accurate and effective classification method can reduce potential financial loss, as well as reduce massive operational cost for the organisations. However due to large amount of unstructured data and high class imbalance, fraud classification is a enormously challenging task. The goal of this paper to find the suitable machine learning model to identify suspicious companies. A data driven approach has been applied to extract features and generate the required dataset. Then different machine learning techniques have been used to handle imbalance classes and build models. based on different performance matrices like ROC, AUC, confusion metrics, mean recall  are used to find the best models and Threshold limits to fine tune to model. \\
    results:\\
    Results of random forest classifier are more accurate than other machine learning algorithm.
    
\end{abstract}