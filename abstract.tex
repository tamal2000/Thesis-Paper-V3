\begin{abstract}
    Trade credit fraud is a major concern for suppliers and insurance companies. An accurate and effective classification method can reduce potential financial loss, as well as reduce massive operational costs for the organisations. However, due to a large amount of unstructured data and high-class imbalance, fraud classification is an enormously challenging task. The goal of this paper is to find a suitable machine learning model to identify suspicious companies. A data-driven approach has been applied to extract features and generate the required dataset. Then different machine learning techniques have been used to handle imbalance classes and build models. based on different performance matrices like ROC, AUC, confusion metrics, recall are used to find the best models and Threshold limits to fine-tune to model. Random Forest classifiers with majority class under-sampling and minority class oversampling by artificial data generated by SMOTE perform the best with the highest AUC score of 0.817. 
    
\end{abstract}