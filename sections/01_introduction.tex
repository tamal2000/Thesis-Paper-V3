% This section includes some motivations behind the work, explicitly or implicitly highlights the research question , provides a high-level explanation of the solution, and describes the contributions.

% establish the context, background and/or importance of the topic
In the current global economy, the financial fraud has become a crucial issue especially for the trade sector. In recent times, the number of fraud cases have increased drastically which effected both financial institutions and their clients significantly. From a survey in 2020 on economic crime and fraud by Price Waterhouse Coopers, its been reported that more than 42 billion USD of financial losses happened in past 24 months~\cite{PwC.Crime.Surveey}. On an average companies reportedly experienced 6 fraud cases during this period. Thirty five percent of fraud cases are marked as Customer fraud.
% brief synopsis of the relevant literature

% Indicate an issue, problem or controversy in the field of study

% listing the reasarch question or hypotheses

% provide synopsis of the research methols

% Significance of value of the study

% Define the topic or key term

% Overview of teh report stucture

% State the purpose of the essay / write

% Provide an overview of the coverage


Some estimates state that fraud costs US business more than \$400 billion annually

Data extracted is a useful method to gather data for analysing and prediction tasks. Data driven is a method base on the history data to build more hyper-parameters to compensate the un-measurable features within the measurable data \cite{SMARRA20181252}. some extending nonlinear features to build the prediction function. In classification and prediction problem, it is essential to discover the pattern of the data and provide us some insights to take some actions according to the prediction results. Random Forest algorithm has been used in handling this problem which are shown effective[4][5] and extensively used in many fields. However it is extremely hard to identify the target information from institutions data based on pattern. The propose of the paper is to find a suitable classifier and its application to find fraudulent cases. 

Random forest generally performs substantially well in cases like this over single tree classifier for imbalance dataset. The target of this paper is to use data-driven method and random forest algorithm in the classification problem to improve the prediction