This section includes some motivations behind the work, explicitly or implicitly highlights the research question , provides a high-level explanation of the solution, and describes the contributions.

Some estimates state that fraud costs US business more than \$400 billion annually 

Data extracted is a useful method to gather data for analysing and prediction tasks. Data driven is a method base on the history data to build more hyper-parameters to compensate the un-measurable features within the measurable data \cite{SMARRA20181252}. some extending nonlinear features to build the prediction function. In classification and prediction problem, it is essential to discover the pattern of the data and provide us some insights to take some actions according to the prediction results. Random Forest algorithm has been used in handling this problem which are shown effective[4][5] and extensively used in many fields. However it is extremely hard to identify the target information from institutions data based on pattern. The propose of the paper is to find a suitable classifier and its application to find fraudulent cases. 

Random forest generally performs substantially well in cases like this over single tree classifier for imbalance dataset. The target of this paper is to use data-driven method and random forest algorithm in the classification problem to improve the prediction