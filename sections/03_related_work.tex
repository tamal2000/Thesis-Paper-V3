
Some of the related studies done by various researchers are highlighted in this section. These researches are on different types of machine learning techniques and various methods for data mining and prepossessing techniques. The main recherche topics are focused on finding fraudulent transactions or even using machine learning methods. 


Asha RB and Suresh Kumar KR proposed a method to detect the fraud in credit card transactions that are based on deep learning \cite{GoodBengCour16} in the paper credit card fraud detection using artificial neural network \cite{RB2021}. Frauds in credit card transactions are the most common and frequent issue. In the paper the researcher usages support vector machine (SVM) \cite{Cristianini2008}, k-nearest neighbour (KNN) \cite{Mucherino2009} and artificial neural network (ANN). The paper concludes that an artificial neural network (ANN) gives an accuracy of 0.99 with a precision of 0.81 and a recall of 0.76.



Class imbalance is one of the major challenges for classifying fraudulent cases. In the paper "Deep Over-sampling Framework for Classifying Imbalanced Data" \cite{ando2017deep} Shin Ando and Chun Yuan Huang have proposed a framework that extends the synthetic over-sampling method to the deep feature space acquired by a convolutional neural network (CNN) \cite{Yamashita2018}. Deep Over-sampling uses the overloaded instances to supplement the minority classes.  


In the paper "Data Mining techniques for the detection of fraudulent financial statements" \cite{KIRKOS2007995}, Efstathios Kirkos, Charalambos Spathis, and Yannis Manolopoulos explore the effectiveness of Data Mining (DM) classification techniques in detecting firms that issue fraudulent financial statements. Data Mining proposes several classification methods derived from the fields of statistics and artificial intelligence. As per the research three methods, which enjoy a good reputation for their classification capabilities


Kamat Nath Mishra and Subhash Chandra proposed a k-fold machine learning techniques for fraud prediction in smart societies \cite{Mishra2021}. In their paper on fraud detection, they have classified the types of credit card fraud types. They proposed a logistic regression-based solution. The implementation of their methodology is then further analysed using machine learning tools like ROC curve \cite{FAWCETT2006861}, confusion matrix, mean-recall score value and precision-recall curve. 


In the paper "Imbalanced Classification Problem Using Data-driven and Random Frost Method" \cite{10.1145/3414274.3414278} the researcher Wan Wang, Xinglu Liu and Victor Chan proposed a classification method for an imbalanced dataset using data-driven technique and random forest. The research is done over three open-source datasets. The result shows that random forest applied with a data-driven approach improved the prediction accuracy. The AUC values also perform well stable and even increased. The researcher augured that they used an imbalanced dataset that represent real-world data. So in theory their approach should perform reasonably in real-world scenarios. 



Important insights for selecting machine learning algorithm for skewed dataset has been discussed in the paper "The relationship between precision-recall and ROC curves" \cite{davis06}. As per the researcher Jass Davis and Mark Goadrich, ROC space and PR space have a deep connection. When a curve dominates in ROC space it also dominates in PR space. They also shared that a model which optimised the area under the ROC curve is not guaranteed to optimise the area under the PR curve. This study can be really handy for our use case of fraud detection. 


In The literature review named "Fraud detection using the fraud triangle theory and data mining techniques" \cite{computers10100121} the researchers highlighted the complexity of predicting Fraud. They explained the human behavioural aspect of fraudulent activity and how traditionally detection was performed by auditors and manual technologies. The researcher mentioned in the paper, in the recent past how several works related to fraud detection using machine-learning techniques were identified without the evidence that they incorporated the fraud triangle as a method for more efficient analysis.