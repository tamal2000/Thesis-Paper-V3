This section provides the necessary context to help the reader understand the remainder of the thesis.

The joint data-driven technique and machine learning algorithms approach are used in this work for fraudulent classification task. Before the classification step, number of new features are generated or engineered from the internal company data. Based on statistical correlation the important features are selected as final features to train various machine learning models. 


Generally the classifiers perform quite weak due to missing values and imbalance dataset for real life scenarios. There are
multiple reasons accounting for this, and one of them is that the
collected data does not contain enough information, because of
the statistical error or some missing values. To obtain more hidden
information, data-driven method could be a good choice. 

Data-driven decision making means getting the right data to the right model to the right time to improve the model for problem solving. This approach can help the organisation to identify and apply recent trend in data to apply it for finding solutions. 


data-driven is a model-free method, which
is not limited by existing features. Part of hidden features can be
found out via data-driven, and the newly updated sample set will
be more likely to achieve better performance. Thus, data driven is
a powerful approach to do feature engineering. We can generate
more new features according to existing features without
destroying the previous ones, and these newly generated features
contain more information than existing ones. [todo: modify]

Firs step of data driven approach is to apply the theory of change. As per the theory of change we can find the road map for how to get company information which are most useful. This method also help to find how to archive long time goals along with indicator for improve and track progress. Then its important to get in touch with internal experts to understand what sort of data they are using for daily tasks. 


